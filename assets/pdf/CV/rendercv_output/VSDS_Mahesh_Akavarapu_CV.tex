\documentclass[12pt, letterpaper]{article}

% Packages:
\usepackage[
    ignoreheadfoot, % set margins without considering header and footer
    top=2 cm, % seperation between body and page edge from the top
    bottom=2 cm, % seperation between body and page edge from the bottom
    left=2 cm, % seperation between body and page edge from the left
    right=2 cm, % seperation between body and page edge from the right
    footskip=1.0 cm, % seperation between body and footer
    % showframe % for debugging 
]{geometry} % for adjusting page geometry
\usepackage[explicit]{titlesec} % for customizing section titles
\usepackage{tabularx} % for making tables with fixed width columns
\usepackage{array} % tabularx requires this
\usepackage[dvipsnames]{xcolor} % for coloring text
\definecolor{primaryColor}{RGB}{0, 35, 90} % define primary color
\usepackage{enumitem} % for customizing lists
\usepackage{fontawesome5} % for using icons
\usepackage{amsmath} % for math
\usepackage[
    pdftitle={VSDS Mahesh Akavarapu's CV},
    pdfauthor={VSDS Mahesh Akavarapu},
    colorlinks=true,
    urlcolor=primaryColor
]{hyperref} % for links, metadata and bookmarks
\usepackage[pscoord]{eso-pic} % for floating text on the page
\usepackage{calc} % for calculating lengths
\usepackage{bookmark} % for bookmarks
\usepackage{lastpage} % for getting the total number of pages
\usepackage{changepage} % for one column entries (adjustwidth environment)
\usepackage{paracol} % for two and three column entries
\usepackage{ifthen} % for conditional statements
\usepackage{needspace} % for avoiding page brake right after the section title
\usepackage{iftex} % check if engine is pdflatex, xetex or luatex

% Ensure that generate pdf is machine readable/ATS parsable:
\ifPDFTeX
    \input{glyphtounicode}
    \pdfgentounicode=1
    \usepackage[T1]{fontenc}
    \usepackage[utf8]{inputenc}
    \usepackage{lmodern}
\fi

\usepackage[default, type1]{sourcesanspro} 

% Some settings:
\AtBeginEnvironment{adjustwidth}{\partopsep0pt} % remove space before adjustwidth environment
\pagestyle{empty} % no header or footer
\setcounter{secnumdepth}{0} % no section numbering
\setlength{\parindent}{0pt} % no indentation
\setlength{\topskip}{0pt} % no top skip
\setlength{\columnsep}{0.15cm} % set column seperation
\makeatletter
\let\ps@customFooterStyle\ps@plain % Copy the plain style to customFooterStyle
\patchcmd{\ps@customFooterStyle}{\thepage}{
    \color{gray}\textit{\small Page \thepage{} of \pageref*{LastPage}}
}{}{} % replace number by desired string
\makeatother
\pagestyle{customFooterStyle}

\titleformat{\section}{
    % avoid page braking right after the section title
    \needspace{4\baselineskip}
    % make the font size of the section title large and color it with the primary color
    \Large\color{primaryColor}
}{
}{
}{
    % print bold title, give 0.15 cm space and draw a line of 0.8 pt thickness
    % from the end of the title to the end of the body
    \textbf{#1}\hspace{0.15cm}\titlerule[0.8pt]\hspace{-0.1cm}
}[] % section title formatting

\titlespacing{\section}{
    % left space:
    -1pt
}{
    % top space:
    0.3 cm
}{
    % bottom space:
    0.2 cm
} % section title spacing

% \renewcommand\labelitemi{$\vcenter{\hbox{\small$\bullet$}}$} % custom bullet points
\newenvironment{highlights}{
    \begin{itemize}[
        topsep=0.10 cm,
        parsep=0.10 cm,
        partopsep=0pt,
        itemsep=0pt,
        leftmargin=0.4 cm + 10pt
    ]
}{
    \end{itemize}
} % new environment for highlights

\newenvironment{highlightsforbulletentries}{
    \begin{itemize}[
        topsep=0.10 cm,
        parsep=0.10 cm,
        partopsep=0pt,
        itemsep=0pt,
        leftmargin=10pt
    ]
}{
    \end{itemize}
} % new environment for highlights for bullet entries


\newenvironment{onecolentry}{
    \begin{adjustwidth}{
        0.2 cm + 0.00001 cm
    }{
        0.2 cm + 0.00001 cm
    }
}{
    \end{adjustwidth}
} % new environment for one column entries

\newenvironment{twocolentry}[2][]{
    \onecolentry
    \def\secondColumn{#2}
    \setcolumnwidth{\fill, 4.5 cm}
    \begin{paracol}{2}
}{
    \switchcolumn \raggedleft \secondColumn
    \end{paracol}
    \endonecolentry
} % new environment for two column entries

\newenvironment{threecolentry}[3][]{
    \onecolentry
    \def\thirdColumn{#3}
    \setcolumnwidth{1 cm, \fill, 4.5 cm}
    \begin{paracol}{3}
    {\raggedright #2} \switchcolumn
}{
    \switchcolumn \raggedleft \thirdColumn
    \end{paracol}
    \endonecolentry
} % new environment for three column entries

\newenvironment{header}{
    \setlength{\topsep}{0pt}\par\kern\topsep\centering\color{primaryColor}\linespread{1.5}
}{
    \par\kern\topsep
} % new environment for the header

\newcommand{\placelastupdatedtext}{% \placetextbox{<horizontal pos>}{<vertical pos>}{<stuff>}
  \AddToShipoutPictureFG*{% Add <stuff> to current page foreground
    \put(
        \LenToUnit{\paperwidth-2 cm-0.2 cm+0.05cm},
        \LenToUnit{\paperheight-1.0 cm}
    ){\vtop{{\null}\makebox[0pt][c]{
        \small\color{gray}\textit{Last updated in June 2024}\hspace{\widthof{Last updated in June 2024}}
    }}}%
  }%
}%

% save the original href command in a new command:
\let\hrefWithoutArrow\href

% new command for external links:
\renewcommand{\href}[2]{\hrefWithoutArrow{#1}{\mbox{\ifthenelse{\equal{#2}{}}{ }{#2 }\raisebox{.15ex}{\footnotesize \faExternalLink*}}}}


\begin{document}
    \placelastupdatedtext
    \begin{header}
        \fontsize{30 pt}{30 pt}
        \textbf{VSDS Mahesh Akavarapu}

        \vspace{0.3 cm}

        \normalsize
        \mbox{{\footnotesize\faMapMarker*}\hspace*{0.13cm}Dept. of CSE, IITK, Kanpur, U.P., India}
        \kern 0.5 cm
        \mbox{\hrefWithoutArrow{mailto:maheshak@cse.iitk.ac.in}{{\footnotesize\faEnvelope[regular]}\hspace*{0.13cm}maheshak@cse.iitk.ac.in}}
        \kern 0.5 cm
        \mbox{\hrefWithoutArrow{https://mahesh-ak.github.io/}{{\footnotesize\faLink}\hspace*{0.13cm}mahesh-ak.github.io}}
        \kern 0.5 cm
        \mbox{\hrefWithoutArrow{https://linkedin.com/in/mahesh-akavarapu}{{\footnotesize\faLinkedinIn}\hspace*{0.13cm}mahesh-akavarapu}}
        \kern 0.5 cm
        \mbox{\hrefWithoutArrow{https://github.com/mahesh-ak}{{\footnotesize\faGithub}\hspace*{0.13cm}mahesh-ak}}
        \kern 0.5 cm
    \end{header}

    \vspace{0.3 cm - 0.3 cm}


    \section{Research Interests}



        
        \begin{onecolentry}
            Computational Linguistics / Natural Language Processing, Historical Linguistics
        \end{onecolentry}

        \vspace{0.2 cm}

        \begin{onecolentry}
            As part of my PhD thesis, I worked on Axial Transformers and Graph-Attention-based architectures over the central problems in historical linguistics incorporating insights from Evolutionary Biology. Additionally, I worked on Phylogenetic testing of language families.
        \end{onecolentry}


    
    \section{Education}



        
        \begin{threecolentry}{\textbf{PhD}}{
            July 2019 to present
        }
            \textbf{Indian Institute of Technology Kanpur}, Computer Science
            \begin{highlights}
                \item GPA: 3.84/4.00 (9.6/10.0)
                \item Advisor: Prof. Arnab Bhattacharya
            \end{highlights}
        \end{threecolentry}

        \vspace{0.2 cm}

        \begin{threecolentry}{\textbf{BT}}{
            July 2014 to May 2019
        }
            \textbf{Indian Institute of Technology Kanpur}, Computer Science, Physics (Second Major)
            \begin{highlights}
                \item GPA: 3.72/4.00 (9.3/10.0)
            \end{highlights}
        \end{threecolentry}


    
    \section{Awards / Honors}



        
        \begin{onecolentry}
            \textbf{Prime Minister's Research Fellowship (2019-24):} Grant of INR 200,000 per year (approx. USD 12,000 in total) from Ministry of Education, Government of India
        \end{onecolentry}

        \vspace{0.2 cm}

        \begin{onecolentry}
            \textbf{Academic Excellence Award from IIT Kanpur (2015):} Awarded to Top 10\% students of the batch
        \end{onecolentry}


    
    \section{Experience}



        
        \begin{twocolentry}{
            Kanpur, India

        Aug. 2019 to July 2024

        4 years 11 months
        }
            \textbf{Indian Institute of Technology Kanpur}, Research Fellow
            \begin{highlights}
                \item Under Prime Minister's Research Fellowship
            \end{highlights}
        \end{twocolentry}


        \vspace{0.2 cm}

        \begin{twocolentry}{
            Kanpur, India

        May 2018 to July 2018

        2 months
        }
            \textbf{Indian Institute of Technology Kanpur}, Student Researcher, Intern
            \begin{highlights}
                \item High Performance Computing in CUDA (Guide: Prof. Mahendra K. Verma)
            \end{highlights}
        \end{twocolentry}



    
    \section{Publications}



        
        \begin{samepage}
            \begin{twocolentry}{
                June 2024
            }
                \textbf{A Likelihood Ratio Test of Genetic Relationship among Languages}

                \vspace{0.10 cm}

                \mbox{V.S.D.S.Mahesh Akavarapu}, \mbox{Arnab Bhattacharya}
                \vspace{0.10 cm}

        Proc. of \textbf{\textit{NAACL 2024}}, Mexico City, Mexico    \end{twocolentry}
        \end{samepage}

        \vspace{0.2 cm}

        \begin{samepage}
            \begin{twocolentry}{
                Mar. 2024
            }
                \textbf{Automated Cognate Detection as a Supervised Link Prediction Task with Cognate Transformer}

                \vspace{0.10 cm}

                \mbox{V.S.D.S.Mahesh Akavarapu}, \mbox{Arnab Bhattacharya}
                \vspace{0.10 cm}

        Proc. of \textbf{\textit{EACL 2024}}, St. Julian's, Malta    \end{twocolentry}
        \end{samepage}

        \vspace{0.2 cm}

        \begin{samepage}
            \begin{twocolentry}{
                Dec. 2023
            }
                \textbf{Cognate Transformer for Automated Phonological Reconstruction and Cognate Reflex Prediction}

                \vspace{0.10 cm}

                \mbox{V.S.D.S.Mahesh Akavarapu}, \mbox{Arnab Bhattacharya}
                \vspace{0.10 cm}

        Proc. of \textbf{\textit{EMNLP 2023}}, Singapore    \end{twocolentry}
        \end{samepage}

        \vspace{0.2 cm}

        \begin{samepage}
            \begin{twocolentry}{
                Jan. 2023
            }
                \textbf{Creation of a Digital Rig Vedic Index (Anukramani) for Computational Linguistic Tasks}

                \vspace{0.10 cm}

                \mbox{V.S.D.S.Mahesh Akavarapu}, \mbox{Arnab Bhattacharya}
                \vspace{0.10 cm}

        Proc. of \textbf{\textit{World Sanskrit Conference 2023}}, Canberra, Australia    \end{twocolentry}
        \end{samepage}


    
    \section{Invited Talks}

    \begin{onecolentry}
        \begin{highlightsforbulletentries}


        \item Evolutionary Biology-Inspired Language Models in Historical Linguistics, University of Tübingen (Online), Apr. 2024

        \item Tutorial on LLMs: Finetuning and Prompting, with Arnab Bhattacharya and Shubham K. Nigam, NIT Warangal (Online), Mar. 2024

        \item Tutorial on Legal Named Entity Recognition, with Shubham K. Nigam, NIT Trichy (Online), Dec. 2022


        \end{highlightsforbulletentries}
    \end{onecolentry}

    \section{Teaching Assitance}

    \begin{onecolentry}
        \begin{highlightsforbulletentries}


        \item (CS 689) \textit{Computational Linguistics for Indian Languages}, Instructor: Prof. Arnab Bhattacharya (Spring 2024, Fall 2022)

        \item (CS 63) \textit{Theory of Computation}, Instructor: Dr. Raghunath Tewari (Fall 2023, Fall 2022)

        \item (CS 315) \textit{Principles of Database Systems}, Instructor: Prof. Arnab Bhattacharya (Spring 2023, Spring 2021)

        \item (CS 657) \textit{Information Retrieval}, Instructor: Prof. Arnab Bhattacharya (Spring 2022)

        \item (CS 771) \textit{Introduction to Machine Learning}, Instructor: Dr. Nisheeth Srivastava (Fall 2021)

        \item (CS 685) \textit{Data Mining}, Instructor: Prof. Arnab Bhattacharya (Fall 2020)

        \item (ESC 101) \textit{Fundamentals of Computing}, Instructor: Dr. Nisheeth Srivastava (Spring 2020), Dr. Piyush Rai (Fall 2019)


        \end{highlightsforbulletentries}
    \end{onecolentry}

    \section{Technologies}



        
        \begin{onecolentry}
            \textbf{Languages:} Python, C/C++, HTML, Shell Scripting etc.
        \end{onecolentry}

        \vspace{0.2 cm}

        \begin{onecolentry}
            \textbf{Libraries:} PyTorch, TensorFlow, HuggingFace, Pandas etc.
        \end{onecolentry}


    

\end{document}